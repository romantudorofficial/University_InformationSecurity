\documentclass[runningheads]{llncs}
\usepackage[T1]{fontenc}
\usepackage{graphicx}
\usepackage{pdfpages}
\usepackage{listings} % Pentru exemple de cod XML
\usepackage{setspace}
\begin{document}
\title{Controlul Accesului pentru XML: Modele, Politici și Implementare}
\author{Biliuți Andrei \and Zară Mihnea-Tudor \and Roman Tudor}
\institute{Universitatea "Alexandru Ioan Cuza" din Iași, Iași IS 700132, România
\email{romantudor.contact@gmail.com}}
\maketitle
\begin{spacing}{1.5}
\begin{abstract}
Această lucrare prezintă o privire de ansamblu cuprinzătoare asupra mecanismelor de control al accesului XML, incluzând modele de securitate, cadre de politici și abordări de implementare. Examinăm diverse modele de control al accesului concepute specific pentru documente XML, discutăm limbaje de specificare a politicilor și analizăm strategii practice de implementare pentru securizarea datelor XML.
\keywords{Securitate XML \and Control al Accesului \and Control al Accesului Bazat pe Roluri \and Criptare XML.}
\end{abstract}



% Introducere

\section{Introducere}

Controlul accesului reprezintă o componentă esențială a securității informațiilor, având scopul de a preveni accesul neautorizat la resursele digitale și de a proteja datele sensibile. În contextul utilizării pe scară largă a limbajului XML pentru reprezentarea și schimbul de date structurate, devine imperativ să implementăm mecanisme de control al accesului care să răspundă provocărilor specifice acestui mediu.

XML facilitează interoperabilitatea între sisteme și aplicații, însă această flexibilitate vine cu riscuri de securitate ridicate. Prin urmare, abordările de control al accesului trebuie să fie precise și granulate, permițând definirea unor politici detaliate care să reglementeze accesul la nivel de element sau atribut XML. De asemenea, modele precum controlul bazat pe roluri (RBAC) și limbajele standardizate, cum ar fi XACML, oferă soluții robuste pentru gestionarea accesului în funcție de context și cerințe specifice.

Această lucrare analizează atât conceptele fundamentale ale controlului accesului pentru XML, cât și aspectele practice legate de implementare, performanță și studii de caz relevante. O atenție deosebită este acordată implementării controlului granular și integrării mecanismelor de aplicare eficiente pentru a minimiza impactul asupra performanței sistemului. Astfel, raportul contribuie la o mai bună înțelegere a metodelor de protecție a datelor în medii complexe, având ca scop final creșterea nivelului de securitate și încredere în aplicațiile moderne.



\section{Modele de Control al Accesului XML}

\subsection{Concepte de Bază}
Scrieți aici despre fundamentele securității XML și bazele controlului accesului.

\subsection{Control al Accesului Fin}
Scrieți aici despre controlul accesului la nivel de element și atribut XML.

\subsection{Controlul Accesului Bazat pe Roluri pentru XML}
Scrieți aici despre implementarea RBAC în contextul XML.

\section{Limbaje pentru Politici de Control al Accesului}

\subsection{XACML}
Scrieți aici despre cadrul XACML și implementare.

\subsection{Specificații de Politici Personalizate}
Scrieți aici despre limbaje specializate pentru politici de securitate XML.

\section{Abordări de Implementare}

\subsection{Mecanisme de Aplicare}
Scrieți aici despre punctele și mecanismele de aplicare a politicilor.

\subsection{Considerații de Performanță}
Gestionarea Listelor de Control al Accesului (ACL) poate reprezenta o provocare computațională costisitoare. Oricând un sistem trebuie să verifice permisiunile pentru părți specifice ale unui fișier XML, precum elemente individuale sau atribute, volumul de muncă poate crește rapid.

Prima modalitate de a rezolva această problemă este prin procesarea bazată pe flux, care rezolvă problemele de memorie într-un mod foarte interesant. În loc să încarce întregul fișier XML în memorie, sistemele verifică permisiunile de acces pe măsură ce datele curg, bucată cu bucată. Este ca și cum ai inspecta obiectele pe o bandă rulantă în mișcare, în loc să le stivuiești pentru a le sorta mai târziu. Mai mult decât atât, multe soluții utilizează indici de salt, care le permit să sară la cele mai importante părți fără a analiza totul.

De asemenea, procesarea bazată pe flux are un avantaj suplimentar în scenariile în care dimensiunea fișierelor XML este foarte mare, reducând semnificativ amprenta memoriei. Spre deosebire de metodele bazate pe Document Object Model (DOM), care necesită încărcarea completă a documentului în memorie, această metodă este ideală pentru aplicații scalabile și eficiente.

Controalele mai granulare pot îmbunătăți securitatea, dar necesită și mai multe resurse, deoarece sistemul trebuie să proceseze mai multe reguli pentru fiecare cerere de acces. Stocarea în cache este, de asemenea, utilizată pentru a îmbunătăți eficiența. Prin salvarea rezultatelor verificărilor anterioare ale permisiunilor, sistemul nu trebuie să refacă aceeași muncă pentru datele accesate frecvent. Este ca și cum un birou de securitate și-ar aminti cine a fost deja autorizat să intre, astfel încât să nu trebuiască să reverifice de fiecare dată.

În plus, implementarea unui mecanism de indexare inteligentă pentru fișierele XML ajută la reducerea timpului de procesare. Indicii hierarhici permit accesul rapid la elementele dorite, în loc să se parcurgă întregul document. Această abordare se dovedește utilă în special pentru bazele de date XML distribuite, unde latențele de acces joacă un rol important.

\section{Studii de Caz}
Un caz interesant provine dintr-un sistem spitalicesc care gestionează fișe medicale electronice. Aceștia au utilizat controlul accesului bazat pe roluri (RBAC) pentru a atribui permisiuni specifice diferitelor roluri din spital. De exemplu, medicii puteau vedea istoricul complet al pacienților, în timp ce asistentele puteau accesa doar detaliile despre medicamente și semnele vitale. Această configurație asigura confidențialitatea pacienților prin restricționarea informațiilor sensibile, permițând în același timp furnizorilor de servicii medicale să obțină datele necesare pentru a-și îndeplini eficient sarcinile.

Un alt exemplu fascinant se ocupă de ceea ce se numește controlul accesului conștient de relații. Aceasta înseamnă protejarea nu doar a punctelor individuale de date, ci și a conexiunilor dintre ele. Imaginați-vă un sistem care gestionează afecțiuni medicale sensibile. Nu este vorba doar despre ascunderea cine are ce afecțiune, ci și despre mascarea legăturilor dintre pacienți și tratamentele lor. Cercetătorii au creat o modalitate de a "masca" anumite relații păstrând în același timp restul datelor utile, ca și cum ai redacta selectiv părți ale unui document păstrând restul lizibil.

În sectorul e-commerce, un alt studiu de caz ilustrează utilizarea controlului accesului pentru gestionarea produselor și comenzilor. Într-un sistem de comerț electronic, anumite roluri, precum managerii de stocuri, aveau acces complet la informațiile despre produse și cantități, în timp ce operatorii de livrări aveau acces doar la datele relevante pentru procesarea comenzilor. Acest model nu doar că a îmbunătățit securitatea, dar a redus și erorile umane prin limitarea accesului inutil la informații.

Al treilea exemplu arată cum controlul accesului pe partea clientului poate face o diferență semnificativă. Inginerii au proiectat un sistem care aplică regulile de acces chiar pe dispozitivul utilizatorului, în loc să se bazeze pe un server central. Această configurație este deosebit de utilă când lățimea de bandă este limitată, deoarece evită trimiterea datelor neautorizate de la bun început. Este ca și cum ai da utilizatorilor propriul paznic miniatura care verifică permisiunile local înainte de a le permite accesul la orice date.

\section{Concluzii}
Controlul accesului pentru documentele XML reprezintă un domeniu complex și în continuă evoluție al securității informației. Prin analiza diverselor modele, politici și implementări prezentate în această lucrare, putem trage următoarele concluzii principale:

În primul rând, importanța controlului granular al accesului în gestionarea documentelor XML nu poate fi subestimată. Capacitatea de a restricționa accesul la nivel de element și atribut oferă flexibilitatea necesară pentru a gestiona informații sensibile în mod eficient, permițând partajarea selectivă a datelor în funcție de necesități.

În al doilea rând, implementările practice demonstrează că există un compromis constant între securitate și performanță. Soluțiile precum procesarea bazată pe flux și utilizarea indicilor de salt oferă modalități promițătoare de a gestiona acest compromis, dar necesită o proiectare atentă și optimizare continuă.

În final, studiile de caz prezentate ilustrează aplicabilitatea largă a controlului accesului XML în scenarii din lumea reală, de la sistemele de sănătate până la gestionarea datelor sensibile în diverse domenii. Aceste exemple subliniază importanța adaptării soluțiilor la cerințele specifice ale fiecărui context de utilizare.

Pe măsură ce complexitatea sistemelor informatice continuă să crească, dezvoltarea unor mecanisme robuste și eficiente de control al accesului pentru XML rămâne un domeniu activ de cercetare și inovare.


\begin{thebibliography}{9}

    \bibitem{damiani2002fine}
    Damiani, E., et al.: A fine-grained access control system for XML documents. ACM Trans. Inf. Syst. Secur. 5(2), 169--202 (2002)
    
    \bibitem{bertino2001authorization}
    Bertino, E., Ferrari, E.: Secure and selective dissemination of XML documents. ACM Trans. Inf. Syst. Secur. 5(3), 290--331 (2002)
    
    \bibitem{oasis2013extensible}
    OASIS: eXtensible Access Control Markup Language (XACML) Version 3.0 (2013)
    
    \bibitem{Evered2002}
    Evered, M., Bögeholz, S.: A Case Study in Access Control Requirements for a Health Information System. \textit{Proceedings of the Australasian Information Security Workshop}, CRPIT Volume 32 (2002). \url{https://crpit.scem.westernsydney.edu.au/confpapers/CRPITV32Evered.pdf}
    
    \bibitem{RoleBasedAccessControl}
    Joshi, A., Joshi, K. P., Finin, T.: Securing XML with Role-Based Access Control: A Case Study in Health Care. In \textit{Information Technology for Management: Emerging Research and Applications}, Springer (2013). \url{https://www.igi-global.com/chapter/securing-xml-with-role-based-access-control/78879}
    
    \bibitem{RelationshipAwareAccessControl}
    Carminati, B., Ferrari, E., Thuraisingham, B. M.: A Rule-Based Approach for Relationship-Aware Access Control for XML Data. \textit{Proceedings of the 30th International Conference on Very Large Data Bases (VLDB)}, Toronto, Canada (2004). \url{https://www.vldb.org/conf/2004/RS3P1.PDF}
    
    \bibitem{ClientBasedAccessControl}
    Dapeng, L., Wei, J.: Client-Based Access Control Management for XML Documents. INRIA (2004). \url{https://inria.hal.science/inria-00070561/document}
    
    \bibitem{spdp2007}
    Ferrari, E., Thuraisingham, B. M., Bertino, E.: Access control and privacy for XML: A review of the state of the art. \textit{Secure Data Management in Decentralized Systems}, Springer (2007). 
    
\end{thebibliography}
\end{spacing}
    
\end{document}
